\documentclass{beamer}
\usetheme{Madrid}
\usecolortheme{dolphin}

\usepackage[utf8]{inputenc}
\usepackage[italian]{babel}
\usepackage{graphicx}
\usepackage{listings}

% Informazioni sul documento
\title{[25-KMChat] Managing Knowledge in Chat with LLMs}
\subtitle{Progetto Digital Transformation}
\author{Alessandro De Martini}
\date{17 Dicembre 2025}

\begin{document}

% Slide Titolo
\begin{frame}
    \titlepage
\end{frame}

% Slide 1: Scenario
\begin{frame}{Scenario e Contesto}
    \begin{block}{Ambito Healthcare}
        Gestione delle terapie domiciliari tramite un assistente virtuale (LLM) a supporto del caregiver (es. infermiere).
    \end{block}
    
    \vspace{0.5cm}
    
    \textbf{Il Flusso di Lavoro:}
    \begin{itemize}
        \item Esiste una \textbf{terapia definita} (knowledge statica/pregressa).
        \item Il caregiver interagisce per \textbf{modifiche giornaliere} (nuova knowledge).
        \item Necessità di gestire eccezioni, urgenze e adattamenti.
    \end{itemize}
\end{frame}

% Slide 2: Requisiti del Sistema
\begin{frame}{Obiettivi e Requisiti Chiave}
    Il sistema deve garantire quattro capacità fondamentali:
    
    \vspace{0.5cm}
    
    \begin{enumerate}
        \item \textbf{Recupero Conoscenza (RAG):} Accedere alle linee guida e alle terapie esistenti.
        \item \textbf{Rilevamento Conflitti:} Identificare incoerenze tra la richiesta e la terapia (es. sovrapposizioni orarie, vincoli medici).
        \item \textbf{Supporto alla Risoluzione:} Segnalare il conflitto al caregiver (Human-in-the-loop), senza decidere autonomamente.
        \item \textbf{Estrazione Conoscenza:} Aggiornare la base dati con le nuove informazioni strutturate (JSON).
    \end{enumerate}
\end{frame}

% Slide 3: La Sfida Tecnica
\begin{frame}{La Sfida: Perché il solo LLM non basta?}
    \begin{alertblock}{Il Problema della Coerenza}
        Gli LLM (Large Language Models) sono probabilistici, non deterministici.
    \end{alertblock}
    
    \vspace{0.5cm}
    
    \begin{itemize}
        \item \textbf{Hallucination:} Un LLM potrebbe "dimenticare" un appuntamento se il contesto è troppo lungo.
        \item \textbf{Calcolo Temporale:} Gli LLM faticano a calcolare sovrapposizioni precise di orari (es. 10:00-11:00 vs 10:45-11:15).
        \item \textbf{Persistenza:} La chat è effimera; le modifiche alla terapia devono essere salvate in modo strutturato.
    \end{itemize}
\end{frame}

% Slide 4: Architettura Proposta
\begin{frame}{Soluzione: Architettura Ibrida (ReAct Agent)}
    Abbiamo implementato un'architettura che separa la \textbf{Semantica} dalla \textbf{Logica}.
    
    \vspace{0.5cm}
    
    \begin{columns}
        \column{0.5\textwidth}
        \textbf{Componente Semantico (LLM)}
        \begin{itemize}
            \item \textbf{Motore:} Llama 3.1 (Locale)
            \item \textbf{Ruolo:} Comprende il linguaggio naturale, estrae intenti, cerca linee guida.
            \item \textbf{Tecnologia:} ChromaDB (RAG).
        \end{itemize}
        
        \column{0.5\textwidth}
        \textbf{Componente Logico (Rule-Based)}
        \begin{itemize}
            \item \textbf{Motore:} Knowledge Manager (Python)
            \item \textbf{Ruolo:} Validazione rigida degli orari, gestione JSON, persistenza.
            \item \textbf{Tecnologia:} Pydantic models.
        \end{itemize}
    \end{columns}
\end{frame}

% Slide 5: Stack Tecnologico
\begin{frame}{Stack Tecnologico}
    Scelte progettuali orientate alla privacy (Ambiente Locale) e all'efficienza.
    
    \vspace{0.5cm}
    
    \begin{description}
        \item[LLM Runtime] \textbf{Ollama} (con modello \texttt{llama3.1:8b}). Privacy totale dei dati sanitari.
        \item[Framework] \textbf{LlamaIndex}. Orchestrazione avanzata per Agenti ReAct.
        \item[Vector DB] \textbf{ChromaDB}. Per indicizzare il PDF e le note non strutturate.
        \item[Data Layer] \textbf{Pydantic \& JSON}. Per la gestione strutturata delle Terapie e Attività.
    \end{description}
\end{frame}

% Slide 6: Stato Avanzamento
\begin{frame}{Implementazione Attuale (Done)}
    \textbf{Step 1: Modellazione Dati}
    \begin{itemize}
        \item Definizione schemi JSON per \texttt{Activity} e \texttt{Therapy} (come da specifiche).
    \end{itemize}
    
    \textbf{Step 2: Knowledge Ingestion}
    \begin{itemize}
        \item Script di caricamento dati su ChromaDB per ricerca semantica.
    \end{itemize}
    
    \textbf{Step 3: Conflict Engine}
    \begin{itemize}
        \item Modulo Python (\texttt{KnowledgeManager}) che rileva sovrapposizioni temporali matematiche.
    \end{itemize}
    
    \textbf{Step 4: Agente Interattivo}
    \begin{itemize}
        \item CLI Chatbot che usa i \textit{Tools} per decidere autonomamente se leggere il calendario o consultare le linee guida.
    \end{itemize}
\end{frame}

% Slide 7: Next Steps
\begin{frame}{Prossimi Passi}
    \begin{itemize}
        \item \textbf{Interfaccia Web:} Porting da CLI a Streamlit per una UX migliore.
        \item \textbf{Conflitti Semantici Avanzati:} Usare l'LLM per rilevare conflitti non temporali (es. interazioni farmaci).
        \item \textbf{Gestione Profili:} Implementare preferenze specifiche per Paziente e Caregiver.
    \end{itemize}
    
    \vspace{1cm}
    \centerline{\Large \textbf{Grazie per l'attenzione}}
\end{frame}

\end{document}
